\subsection{Variables and Measurement}
Choosing the right statistical tools for a test setup requires that one considers statistical methods and variables before even initiating the design. Basically it can be said that there are four types of variables to consider:

\begin{itemize}
  \item Dependent variables
  \item Independent variables
  \item Discrete variables
  \item Continuous variables
\end{itemize}

Furthermore one must consider the levels of variables to be measured and analyzed in tests. Therefore it is important to keep in mind from the beginning which variables makes sense to analyze, and how the results can become useful. Eventually one collects data to understand how systems work, how products can be improved or how objects or actions causes change (causality) in e.g. defined environments. The fundamental understanding of these analytical terms will be shortly described in the following chapter.\\\\
\textbf{Dependent variables} are the values that are not being manipulated. They can also be described as the “outcome variables” \citep[page 21]{Design}. The measured values are dependant of of the other values in the system.\\\\
\textbf{Independent variables} are the values which can be said to be the “input variables”, which are the values that are being manipulated in a test environment. These variables are defined by the experimenter.\\\\
\textbf{Discrete variables} can be described as Boolean logic. The typical example for discrete variables is the pregnancy example on whether a woman is pregnant or not pregnant, and nothing in between \citep[page 9]{Design}.\\\\
\textbf{Continuous variables} can be scalable values that describes to what degree an event occurs or how much or how little something can be observed. Ratings and estimations are often defined as ratios. 
\\\\One must consider the above mentioned variable types before proceeding to the next step, that be the choice of data variables to place measurements in and to perform statistical analysis on. But before even considering data variable types, one must also distinguish between the two main categories described as: 
\begin{itemize}
\item Parametric
\item Non-parametric
\end{itemize}
\textbf{Parametric data} is considered to be the more powerful than its non-parametric equivalent \citep[page 21]{Design}, as it can be described as being: normally-distributed, based on arithmetic measurements of the continuous interval/ratio variable. The measurements are compareable as they represent concrete values measured by instruments or ratings.
Parametric Data Variables:
There are two fundamental data variables belonging to the parametric data types, which are described in the following chapter:
\begin{itemize}
\item Interval
\item Ratio 
\end{itemize}
\textbf{Interval data} are considered to be a preferred datatype from the ordinal data, and are also being utilized in psychological fields of study and research \citep[page 8]{Design}. Interval data are defined as being collected in an interval scale e.g used to describe feelings, moods and attitudes towards clearly specified questions, that be attempts to rate anxiety to scary animals. Interval data are often specified into a 10-point scale with equal values dividing the steps. These steps does not indicate specific values but solely intervals, which means that the numbers in the interval are not necessarily real numerical values.\\\\
\textbf{Ratio data} are said to have the properties of interval data, but contains improved scalability and reveals differences into details. E.g. when measuring temperatures or to compare errors made by test subjects. Ratio data are particularly ideal for studies and research that require nuanced figures.\\\\
\textbf{Non-parametric data} are known to be free of assumptions as they do not meet the criterias contained in parametric data. That being said, non-parametric data contains elements of data with the characteristics of parametric data but are collected more randomly and commonly visualized into histograms. Typically collected data are analyzed and ranked, which can result in loss of detailed informations. 
\subsubsection{Non-parametric Data Variables:}
\begin{itemize}
\item Nominal
\item Ordinal
\end{itemize}
\textbf{Nominal data} are solely representing names for segments, players or numbers. They are not representing actual values but can be described as numerical semiotics e.g. in soccer the goal keeper is always playing with the number “1” and the attackers “10 \& 11”. Nominal data can not be considered as arithmetic but instead this type of data can be utilized to compare simular data segments for tests performed on e.g. pears or apples.\\\\
\textbf{Ordinal Data} are most oftenly used to describe when events occur e.g. who is first, second and third. This type of data does not indicate any other value and does not describe nuances nor important differences in specific values. Therefore this data type is not considered to be particularly useful as the most important information often lies in the details. 
\subsubsection{Statistical Tools}
With the above mentioned levels of data variables and types taken into consideration it is possible to make qualified choices of test scenarios and statistical methods. Whether analysis should be based on frequencies or scores, independent/dependant variables, experimental or correlational tests, parametric or non-parametric, groups and conditions, it is important to identify the purpose of the measurement and analysis. Considering these aspects one must choose between Chi-square, Pearsons, Spearmans, Anova, t-test etc. in order to collect the most useful results out of the performed tests with all the possible variables taken into consideration. For example, when having only one independent variable one would utilize the Chi-square, but when dealing with parametric or non-parametric data, one would choose between Pearson’s r and Spearman’s rho. It is of fundamental importance to consider these aspects as early as possible in order to understand and support the potential value of the tests \citep[page 274]{Design}.
