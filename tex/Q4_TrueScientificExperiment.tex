\subsection{True Scientific Experiment}
There are different elements determine our experiment are a true scientific experiment, in short elements are that our experiment are: measurable, testable, analyzable, falsifiable hypotheses and our experiment can be repeated.
In DAEBOOK XX REF Karl Popper talks about the difference between a scientific statement and a non-scientific statement. A non-scientific statement could sound like this: “Guys look better with makeup”. A statement like this is according to XX DEABOOK non-scientific, since it cannot be proven or disproved. Although most men do not wear makeup, and therefore one might claim the statement false. But since it cannot be tested and therefore only be judged by people subjectively. 
A scientific statement, on the other hand, can be tested and thereby confirmed or disconfirmed the statement as mentioned in XX DAEBOOK page 17. An example on a scientific statement could be like this: “Humans cannot without diver equipment breath under water “.
In our report the statement we used for our null hypothesis: Two knn-classifiers with a different k and otherwise identical, will not result in significantly different confusion tables. 

The theory behind falsifiability 
