\subsection{True Scientific Experiment}
There are different elements that determine if an experiment are a true scientific experiment, in short elements are that an experiment are: measurable, testable, analyzable, falsifiable hypotheses and that the experiment can be repeated.

In \citep{Design} the author talks about how Karl Popper see the difference between a scientific statement and a non-scientific statement. A non-scientific statement could sound like this: “Guys look better with makeup”. A statement like this is according to \citep{Design} non-scientific, since it cannot be proven or disproved. Although most men do not wear makeup, and therefore one might claim the statement false. But since it cannot be tested and therefore only be judged by people subjectively. 
A scientific statement, on the other hand, can be tested and thereby confirmed or disconfirmed the statement as mentioned in \citep{Design} page 17. An example on a scientific statement could be like this: “Humans cannot without diver equipment breath under water“.
In our report the statement we used for our null hypothesis: "Two knn-classifiers with a different k and otherwise identical, will not result in significantly different confusion tables". 

This is a testable hypothesis, since both independent and dependent variables are clearly defined, which thereby also makes it a scientific hypothesis. But as mentioned earlier, there are more to why it is a true scientific experiment. 


Our hypothesis follows the falsification rule.
The falsification rule is when one try to disconfirm a hypothesis instead of trying to confirm the hypothesis. The method tells just as much as when one tries to confirm the hypothesis. Here is an example to help explaining  falsification. If our hypothesis sounded like this: all cats hate humans. If then one cat did not hate humans, then even if it is only one cat then the theory is disconfirmed. As Field mentions in \citep{Design} "one instance that disconfirms a hypothesis is more powerful than many instances that confirm the hypothesis".

There are different kinds of hypothesis, one is called the experimental/alternative hypothesis, and another is called the null hypothesis. As explained in the \citep{Design} page 141 The experimental hypothesis predicts that your experiment will have an effect, where the null hypothesis prediction is wrong and there is no effect.
We use the null hypothesis to confirm or reject our experimental hypothesis. According to the \citep{Design} the experimental hypothesis or the null hypothesis can never be completely true, which is why when working with confirming or disconfirming experimental and null hypothesis, you work in percentages. For this percentage, Fisher suggest for a threshold for confidence we should use 95\% meaning that the experimental hypothesis is accepted when there is a 95\% certainty that the results are genuine. Bonferronis correction was applied, which is explained further when explaining the test results. 

When making a scientific experiment you are working with the manipulation of variables \citep{Design} page 21. The variables that are manipulated, are what is called the independent variables \citep{Design} 
Then there are the dependent variables, their value depends on the rest of the variables in the experiment. The different variables is what John Stuart Mill proposed to be used to find casual factors. These could be found by comparing two conditions: one with supposed cause present, and one where it is not \citep{Design}.