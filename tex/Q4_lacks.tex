\section{General Bias (what was lacking or what was less reliable/valid)}
%introduction
Here we will discus possible biases in our project, with focus on what biases that could effect the validity and reliability of our evaluation. We will first off discus how the data that was used in the project was collected, how the condition the collection took place and how the segmentation of the data for our program could have effected the results of the final evaluation of the data. Then we will discus what data was the output from the program will be shortly presented, then other methods of getting better data (we got mainly nominal data / frequency data) will be discussed.

%how was the data collected, problems with the data set e.g noise, how it was segmented and target group. also manual seg vs auto seg.
The data collected for this project was music files that contains Beatboxing from different participants. the participants were primary non beatboxers, they was found using convenience sampling method, which means that the participants was chosen based on proximity,  the fact that we did not consider their ability of beatboxing could contribute to making the validity less, because the way that the participants made beatboxing sounds made might not be correct to the sound that beatboxers with experience make. The collection took place on AAU campus Copenhagen at ac meyers vænge 15 in the main hall, the location of taken the main hall can contribute to more noise in the recordings which might will make the later classification different from recording at other places with less or more noise. next after the collection of the dataset we need to segment the audio to make a sound segments that contain the sound of beatboxing or noise, this was done in two way, one where it was eye balled to find the segments(using sonicvisualiser) and one where there was made use of a feature to find the sound segments(RMS). The segments that the two methods could be different in precision of how closely the segmentation can be (using sonic visualiser one can see on the waveform where the sound begins which can make a close segmentation, the RMS is taken in windows because of that one might get an overshoot of the sound), this can influence the classification of the segment sound section because one might contain more information than just the sound that we want to analyse. These different bias might contribute to validity of the later classification of the sound that then again effect the evaluation that we later did (chi squared).

%our test what we could have tested differently (we got frequency data any way that we could have gotten some kindof scale data instead perhaps with a test of executions speed, going out and make people try it to get there input into how correct they felt that their beatboxing was transcribed)

for the evaluation of our program matrices was made that shows the accuracy of the classifier with various settings and different features, secondly to make a statistical evaluation of the results we made chi squared test. The reason we used a chi squared test was that the data we collected from the classification program, the matrices, are categorized and they by that they are nominal data. To change that instead of focussing on how accurate the system is we could make a test of how fast the program is with the different features, this will give us different timings instead and this will give us ratio data which can give us more opportunities on what test we can use. A test that could be used with these different data will answer another question than the one we ask, it will be more of a measure of what is fastest and not the best way. we chose to use the first test because it would give a better picture of how the system performs with the different classifier.
%sum up


to sum up there are a few changes that could be made to make the experiment more valid, but the overall validity of the experiment can be considered to be sufficient. For the reliability we can say that the experiment should be possible to recreate if a dataset like our was collected or if ours dataset was provided, there would properly be some deviation in the result in the level of at few percentage in how accurately the classes will be determined, because of different implementation.