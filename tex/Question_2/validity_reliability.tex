\hspace{0pt} \\
\\
Reliability and validity are two fundamental aspects of any experimental design. The idea and purpose behind these are to control the systematic measurement errors, and thereby reducing the inconsistency of the measurement \citep[p. 44-48]{Design}. The aim of importance is simply whether a study or research is of importance. That is, however, a rather subjective matter. Time also changes what is found as of importance. If research is neither valid nor reliable, it cannot possibly be of importance \citep[p. 54]{Design}.

Reliability refers to the consistency of a measure. For a measure to be reliable, it must be replicable and produce the same results under same conditions. There are several ways to estimate reliability of a measurement instrument. \textit{Test-retest} reliability is the ability to repeat the measurements across different points in time. This type of reliability wants to achieve the same results each time \citep[p. 47]{Design}. Another method to measure the consistency (reliability) is the \textit{split-half} method. E.g. in a questionnaire consisting of 10 items, the method splits it into two groups of 5 items. For each half, a score is calclulated, and then the two halves are compared. If they are reliable, the two halves should have a large correlation. \textit{Cronbach’s alpha} is the average of split correlations, and an alpha value of 0.8 or above is accepted to be reliable \citep[p. 48]{Design}.

Validity is essentially whether the measurements actually measure what it is supposed to, or what it claims to measure. There are many types of validity. One of them is \textit{content validity}. The content of a test should cover the content of the construct it was designed for. It is obtained by carefully selecting the “right” items to be included in a test \citep[p. 44-46]{Design}. Another type of validity is \textit{criterion validity}, which in essence is whether a measurement accurately predicts an outcome measure, e.g. by correlating scores with other known measures of the construct \citep[p. 46-47]{Design}. A third type of validity is \textit{factorial validity}. It is based on factorial analysis, which is a statistical technique to find out which items, or questions in a questionnaire, that are related. It is used when designing proper questions for a questionnaire, where sub-components should make intuitive sense and this is where factorial validity can be used \citep[p. 47]{Design}. Other types of validity exist, but are not included in this mini-project. However, the two most common forms of validity is \textit{internal} and \textit{external validity}. The internal validity concerns following the principles of cause and effect, which is evidence that the manipulation of the independent variable had an effect on the outcome result. As said, the purpose of validity is to minimize and control the potential errors, and there are many factors that can threat the internal validity. Eight of the most common threats to internal validity are as follows \citep[p. 58-62]{Design}:

\begin{itemize}
\item \textbf{Group threats}: differences in groups at the start of the study .
\item \textbf{Regression to the mean}: participants produce very high/very low scores by chance.
\item \textbf{Time threats}: participants' behavior may change over time.
\item \textbf{History}: unrelated to the manipulation of the independent variable, events can occur simultaneously with the experiment, and it becomes and alternative explanation to the change in participants’ behavior. 
\item \textbf{Maturation}: participants’ behavior may change due to natural development (i.e. children). 
\item \textbf{Instrument change}: changes due to change of instrument, e.g. an interviewer gets more tired or bored during the day, or maybe becomes more excited. 
\item \textbf{Differential mortality}: participants may drop-out, making it hard to compare pre- and post-tests, and leaving behind participants’ with systematic differences.
\item \textbf{Reactivity and Experimenter Effects}: measuring participants’ behavior may affect their behavior. The experimenter can bias the results in the way they interact with the participants. Can be minimized using the “double-blind” technique, in which the experimenter and participants are unaware of the experimental hypothesis.
\end{itemize} 

The external validity is more a question of how the measurements can be generalized. Here can be mentioned two types of threats to the external validity, which reduces the generalization \citep[p. 62]{Design}. The first threat is \textit{over-use of special participant groups}, e.g. when the participants are voulenteers, they might be more engaded in the study than non-voulenteers, and so they do not have same characteristics as the general population \citep{Lard}. The second is \textit{restricted number of participants}, where experiments often use too few participants to attain statistical significance, which is in particular also a thread to the reliability \citep[p. 62]{Design}.

To maximize the reliability, experimenters should describe specific definitions on what is being measured \citep[p. 57]{Design}. Isolating causal factors, and minimize alternative explanations will increase the validity \citep[p. 62]{Design}. Running pilot tests before the real data-collection, as well as having experts reviewing the content, can reveal some errors and problems which can be adjusted to increase the reliability and validity of a study \citep[p. 407-408]{ResearchMethods} \citep{NCTI}. Many potential unsystematic variations can be eliminated by randomization, in which participants are randomly allocated in the parts of the study \citep[p. 24]{Design}.