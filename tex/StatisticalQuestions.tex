\subsection{Question 1: Measurement}
\textit{Discuss levels on which variables could be measured. How the level of measurement influences the choice of statistical tools? (Hint: While answering this question, consult Chapter 8 of the book).}

\subsection{Question 2: Experimental Design}
\textit{In an experimental design, what do aims Reliability, Validity and Importance mean? Which actions should be taken to maximize measurements' reliability and validity?}

(1) Should a one or two-tailed test be used if we want to know if there is a significant difference in gaming time between the two types of feedback? How many degrees of freedom?

\textbf{ANSWER:} Two-tailed. We're looking for a difference, i.e. non-specific. Thus the null-hypothesis states that there's no difference between the two types of feedback. One tail says that auditory feedback gives the best result while the other tail says visual feedback is best.

Degrees of freedom is 18 (2 test with 10 results for each [2*(10-1)])

(2) Use the correct t-test and answer the question whether there is a significant difference (=.05).

\textbf{ANSWER:} Use a dependant t-test, as each participants has been used in both tests (auditory & visual). First test for homogeneity of variance (kstest / smirnof).

\subsection{Question 3: Analysis of Presented Data}
\textit{Analyze and present data gathered by a group of MED10 students for their semester project. Use available materials as a guideline and inspiration. Follow the document DAEV-ANOVA-exercises.pdf, and be sure to make all statistical tests mentioned there and to answer each of the points. Do all needed calculations and make all illustrative figures in Matlab.}